\chapter{Решение задачи}
\section{Целевая функция}
\vspace*{-1em}
\quad\\
$\letus R \in \R_+$ -- радиус окружности;\\
\phantom{t} Начало координат находится в центре окружности;\\
\phantom{t} $[-R; R] \ni x$ -- точка на оси абсцисс.
\begin{lemma}\label{lem:1}\quad\\
    На хорде, перпендикулярной оси абсцисс и проходящей через точку $(x, 0)$, поместится ровно $f(x, h)$ целых отрезков длины $h \in \R, 0 < h < \sqrt{2}R$.
    $$f(x, h) = \lfloor \frac{2\cdot\sqrt{R^2 - x^2}}{h} \rfloor$$
\end{lemma}
$\letus \frac{\sqrt{4R^2-h^2}}{2} \le r \le R$\\
$\rho$ -- разбиение отрезка $[-r, R]$ с шагом $h$
\begin{lemma}\label{lem:2}\quad\\
    На отрезке $\rho \ni [x_0, x_1]$ оси абсцисс в окружность поместится ровно $g(x_0, x_1, h)$ квадратов со стороной $h$.
    \[
    g(x_0, x_1, h) = 
    \begin{cases}
        f(x_0, h), &R - |x_0| \le R - |x_1|\\
        f(x_1, h), &R - |x_0| > R - |x_1|\\
    \end{cases}
    \]  
\end{lemma}
$\letus K \in \N$ -- необходимое количество квадратов;
\begin{definition}
    Нижним псевдо-интегралом Дарбу назовем:
    $$\large
        M_*(h, \rho) = \sum_{[x_0, x_1] \in \rho}g(x_0, x_1, h)
    $$
\end{definition}
\begin{definition} Целевой функцией назовем:
    $$M(h, \rho) = 
    \begin{cases}
        h^2K, &M_*(h, \rho) \ge K\\
        0, &M_*(h, \rho) < K
    \end{cases} $$
\end{definition}

Заметим, что ген $\rho$ представим в виде вещественного числа (первая точка разбиения). 
Таким образом, представители популяции являются носителями двух вещественнозначных генов. Задача сводится к максимизации целевой функции при заданных условиях.
\section{Изменчивость}
$\letus$ Ген $t$ у особи $p$ мутирует с некоторой вероятностью $P_m$
\begin{definition} Значение $t$ после мутации $t^*$ определяется, как нормальная случайная величина с средним $t$ и стандартным отклонением $\sigma$, ограниченная допустимыми значениями гена
\end{definition}

$\letus$ Особь $p$ становится родителем с некоторой вероятностью $P_c$
\begin{definition}
    $\letus$ Выбраны два родителя $p_1, p_2; \alpha \ge 0$, тогда интервалы допустимых значений генов для их потомков определяются следующим образом:
    {\large \[
    T_t = [\max\{t_{\min}, p_{1_{t}} - \alpha(p_{2_{t}} - p_{1_{t}})\}; \min\{t_{\max}, p_{2_{t}} + \alpha(p_{2_{t}} - p_{1_{t}})\}]
    \]}
\end{definition}
\begin{definition}
    Потомок определяется, как $p(p_1, p_2) := (h \in T_h(p_1, p_2), \rho \in T_{\rho}(p_1, p_2))$ (гены выбираются случайно).
\end{definition}
\begin{algo}
$\letus$ Есть список особей из популяции, которые становятся родителями, тогда
родители попарно скрещиваются, и из каждой пары в популяцию добавляется $c \in \N$ потомков.    
\end{algo}
\section{Отбор}
$\letus N$ -- размер популяции в каждом новом поколении.\\ 
После скрещивания и мутаций особи ранжируются по значению целевой функции. 
В новое поколение проходит $N^* = \lfloor nN \rfloor (0 \le n \le 1)$ лучших особей. 
Далее происходит турнирный отбор $N - N^*$ особей (по $m \in \N$ представителей в раунде). $N$ отобранных особей выживают, остальные погибают.

Начальная популяция формируется из комбинаций $\lfloor \sqrt{N} \rfloor$ равномерно распределенных значений $h$ и $\rho$, итого $\thickapprox N$ особей в популяции, недостающие особи выбираются случайными допустимыми значениями генов.

Смена поколений происходит следующим образом: скрещивание $\rightarrow$ мутация $\rightarrow$ отбор.
Алгоритм совершает $D \in \N$ итераций ($D$ поколений), но если на протяжении $E \in \N$ поколений наилучшее значение целевой функции изменяется не более чем на $\varepsilon > 0$, то алгоритм досрочно завершает свою работу.
