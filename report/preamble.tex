\usepackage[utf8]{inputenc}

\usepackage[T2A,T1]{fontenc}

\usepackage[english,russian]{babel}

\usepackage{textcomp}

\usepackage{url}
\usepackage{hyperref}

\hypersetup{
    colorlinks,
    linkcolor={black},
    citecolor={black},
    urlcolor={blue!80!black}
}

\usepackage{graphicx}

\usepackage{float}

\usepackage{booktabs}

\usepackage[shortlabels]{enumitem}

\usepackage{emptypage}

\usepackage[usenames,dvipsnames]{xcolor}

\usepackage{amsmath, amsfonts, mathtools, amsthm, amssymb}

\usepackage{mathrsfs}

\usepackage{cancel}

\usepackage{bm}

\usepackage{geometry}
\geometry{top=72pt}
\geometry{bottom=72pt}
\geometry{left=72pt}
\geometry{right=72pt}


\newcommand\N{\ensuremath{\mathbb{N}}}
\newcommand\R{\ensuremath{\mathbb{R}}}
\newcommand\Z{\ensuremath{\mathbb{Z}}}
\renewcommand\O{\ensuremath{\emptyset}}
\newcommand\Q{\ensuremath{\mathbb{Q}}}
\renewcommand\C{\ensuremath{\mathbb{C}}}

\DeclareMathOperator{\sgn}{sgn}

\newcommand{\tabitem}{%
  \usebeamertemplate{itemize item}\hspace*{\labelsep}}

\usepackage{systeme}

\usepackage{indentfirst}
%\setlength{\parindent}{1.5cm}

\usepackage{xcolor}       % Пакет для работы с цветами
\usepackage{listings}     % Пакет для вставки кода
\lstset{
  language=Python,                      % Язык по умолчанию
  backgroundcolor=\color{black!5},      % Легкий серый фон
  basicstyle=\ttfamily,           % Шрифт (моноширинный, мелкий)
  keywordstyle=\color{blue},            % Стиль для ключевых слов (if, for, etc.)
  stringstyle=\color{red!70!black},     % Стиль для строк ("hello")
  commentstyle=\color{green!70!black},  % Стиль для комментариев (# comment)
  numberstyle=\color{black},        % Стиль для номеров строк
  breaklines=true,                      % Автоматически переносить длинные строки
  showstringspaces=false,               % Не показывать специальные символы для пробелов в строках
  numbers=left,    
  xleftmargin=15pt                     % Нумерация строк слева
}

\def\letus{%
    \mathord{\setbox0=\hbox{$\exists$}%
             \hbox{\kern 0.125\wd0%
                   \vbox to \ht0{%
                      \hrule width 0.75\wd0%
                      \vfill%
                      \hrule width 0.75\wd0}%
                   \vrule height \ht0%
                   \kern 0.125\wd0}%
           }%
}

\usepackage{tikz}
\usepackage{tikz-cd}

\usetikzlibrary{intersections, angles, quotes, calc, positioning}
\usetikzlibrary{arrows.meta}

\usepackage{pgfplots}
\pgfplotsset{compat=1.13}

\usepackage{thmtools}
\usepackage[framemethod=TikZ]{mdframed}
\mdfsetup{skipabove=1em,skipbelow=0em, innertopmargin=5pt, innerbottommargin=6pt}

\theoremstyle{definition}

\makeatletter

\@ifclasswith{report}{nocolor}{
    \declaretheoremstyle[headfont=\bfseries, bodyfont=\normalfont, mdframed={ nobreak } ]{thmgreenbox}
    \declaretheoremstyle[headfont=\bfseries, bodyfont=\normalfont, mdframed={ nobreak } ]{thmredbox}
    \declaretheoremstyle[headfont=\bfseries, bodyfont=\normalfont]{thmbluebox}
    \declaretheoremstyle[headfont=\bfseries, bodyfont=\normalfont]{thmblueline}
    \declaretheoremstyle[headfont=\bfseries, bodyfont=\normalfont, numbered=no, mdframed={ rightline=false, topline=false, bottomline=false, }, qed=\qedsymbol ]{thmproofbox}
    \declaretheoremstyle[headfont=\bfseries, bodyfont=\normalfont, numbered=no, mdframed={ nobreak, rightline=false, topline=false, bottomline=false } ]{thmexplanationbox}
    \AtEndEnvironment{eg}{\null\hfill$\diamond$}%
}{
    \declaretheoremstyle[
        headfont=\bfseries\color{ForestGreen!70!black}, bodyfont=\normalfont,
        mdframed={
            linewidth=2pt,
            rightline=false, topline=false, bottomline=false,
            linecolor=ForestGreen, backgroundcolor=ForestGreen!5,
        }
    ]{thmgreenbox}

    \declaretheoremstyle[
        headfont=\bfseries\color{NavyBlue!70!black}, bodyfont=\normalfont,
        mdframed={
            linewidth=2pt,
            rightline=false, topline=false, bottomline=false,
            linecolor=NavyBlue, backgroundcolor=NavyBlue!5,
        }
    ]{thmbluebox}

    \declaretheoremstyle[
        headfont=\bfseries\color{NavyBlue!70!black}, bodyfont=\normalfont,
        mdframed={
            linewidth=2pt,
            rightline=false, topline=false, bottomline=false,
            linecolor=NavyBlue
        }
    ]{thmblueline}

    \declaretheoremstyle[
        headfont=\bfseries\color{RawSienna!70!black}, bodyfont=\normalfont,
        mdframed={
            linewidth=2pt,
            rightline=false, topline=false, bottomline=false,
            linecolor=RawSienna, backgroundcolor=RawSienna!5,
        }
    ]{thmredbox}

    \declaretheoremstyle[
        headfont=\bfseries\color{RawSienna!70!black}, bodyfont=\normalfont,
        numbered=no,
        mdframed={
            linewidth=2pt,
            rightline=false, topline=false, bottomline=false,
            linecolor=RawSienna, backgroundcolor=RawSienna!1,
        },
        qed=\qedsymbol
    ]{thmproofbox}

    \declaretheoremstyle[
        headfont=\bfseries\color{NavyBlue!70!black}, bodyfont=\normalfont,
        numbered=no,
        mdframed={
            linewidth=2pt,
            rightline=false, topline=false, bottomline=false,
            linecolor=NavyBlue, backgroundcolor=NavyBlue!1,
        },
    ]{thmexplanationbox}
}

\declaretheorem[style=thmgreenbox, numbered=no, name=Определение]{definition}
\declaretheorem[style=thmbluebox, numbered=no, name=Пример]{eg}
\declaretheorem[style=thmredbox, numbered=no, name=Задача]{ex}
\declaretheorem[style=thmredbox, numbered=no, name=Утверждение]{prop}
\declaretheorem[style=thmredbox, numbered=no, name=Теорема]{theorem}
\declaretheorem[style=thmredbox, numbered=no, name=Лемма]{lemma}
\declaretheorem[style=thmredbox, numbered=no, name=Следствие]{corollary}
\declaretheorem[style=thmredbox, numbered=no, name=Алгоритм]{algo}

\@ifclasswith{report}{nocolor}{
    \declaretheorem[style=thmproofbox, name=Доказательство]{replacementproof}
    \declaretheorem[style=thmexplanationbox, name=Доказательство]{explanation}
    \declaretheorem[style=thmproofbox, name=Решение]{solution}
}{
    \declaretheorem[style=thmproofbox, name=Доказательство]{replacementproof}
    \renewenvironment{proof}[1][\proofname]{\vspace{-10pt}\begin{replacementproof}}{\end{replacementproof}}

    \declaretheorem[style=thmexplanationbox, name=Доказательство]{tmpexplanation}
    \newenvironment{explanation}[1][]{\vspace{-10pt}\begin{tmpexplanation}}{\end{tmpexplanation}}
}

\makeatother

\declaretheorem[style=thmblueline, numbered=no, name=Замечание]{remark}
\declaretheorem[style=thmblueline, numbered=no, name=Примечание]{note}

\usepackage{etoolbox}
\AtEndEnvironment{vb}{\null\hfill$\diamond$}%
\AtEndEnvironment{intermezzo}{\null\hfill$\diamond$}%

\usepackage{fancyhdr}
\pagestyle{fancy}

\fancyhead[RO,LE]{Реализация генетических алгоритмов с использованием GUI}
\fancyhead[RE,LO]{}
\fancyfoot[LE,RO]{\thepage}
\fancyfoot[C]{\leftmark}

\makeatother

\usepackage{import}
\usepackage{xifthen}
\pdfminorversion=7
\usepackage{pdfpages}
\usepackage{transparent}

\newcommand{\incfig}[1]{%
    \def\svgwidth{\columnwidth}
    \import{./figures/}{#1.pdf_tex}
}

\pdfsuppresswarningpagegroup=1

\newcommand{\nchapter}[2]{%
    \setcounter{chapter}{#1}%
    \addtocounter{chapter}{-1}%
    \chapter{#2}
}

\newcommand{\nsection}[3]{%
    \setcounter{chapter}{#1}%
    \setcounter{section}{#2}%
    \addtocounter{section}{-1}%
    \section{#3}
}%



